% Options for packages loaded elsewhere
% Options for packages loaded elsewhere
\PassOptionsToPackage{unicode}{hyperref}
\PassOptionsToPackage{hyphens}{url}
\PassOptionsToPackage{dvipsnames,svgnames,x11names}{xcolor}
%
\documentclass[
  letterpaper,
  DIV=11,
  numbers=noendperiod]{scrartcl}
\usepackage{xcolor}
\usepackage{amsmath,amssymb}
\setcounter{secnumdepth}{-\maxdimen} % remove section numbering
\usepackage{iftex}
\ifPDFTeX
  \usepackage[T1]{fontenc}
  \usepackage[utf8]{inputenc}
  \usepackage{textcomp} % provide euro and other symbols
\else % if luatex or xetex
  \usepackage{unicode-math} % this also loads fontspec
  \defaultfontfeatures{Scale=MatchLowercase}
  \defaultfontfeatures[\rmfamily]{Ligatures=TeX,Scale=1}
\fi
\usepackage{lmodern}
\ifPDFTeX\else
  % xetex/luatex font selection
\fi
% Use upquote if available, for straight quotes in verbatim environments
\IfFileExists{upquote.sty}{\usepackage{upquote}}{}
\IfFileExists{microtype.sty}{% use microtype if available
  \usepackage[]{microtype}
  \UseMicrotypeSet[protrusion]{basicmath} % disable protrusion for tt fonts
}{}
\makeatletter
\@ifundefined{KOMAClassName}{% if non-KOMA class
  \IfFileExists{parskip.sty}{%
    \usepackage{parskip}
  }{% else
    \setlength{\parindent}{0pt}
    \setlength{\parskip}{6pt plus 2pt minus 1pt}}
}{% if KOMA class
  \KOMAoptions{parskip=half}}
\makeatother
% Make \paragraph and \subparagraph free-standing
\makeatletter
\ifx\paragraph\undefined\else
  \let\oldparagraph\paragraph
  \renewcommand{\paragraph}{
    \@ifstar
      \xxxParagraphStar
      \xxxParagraphNoStar
  }
  \newcommand{\xxxParagraphStar}[1]{\oldparagraph*{#1}\mbox{}}
  \newcommand{\xxxParagraphNoStar}[1]{\oldparagraph{#1}\mbox{}}
\fi
\ifx\subparagraph\undefined\else
  \let\oldsubparagraph\subparagraph
  \renewcommand{\subparagraph}{
    \@ifstar
      \xxxSubParagraphStar
      \xxxSubParagraphNoStar
  }
  \newcommand{\xxxSubParagraphStar}[1]{\oldsubparagraph*{#1}\mbox{}}
  \newcommand{\xxxSubParagraphNoStar}[1]{\oldsubparagraph{#1}\mbox{}}
\fi
\makeatother


\usepackage{longtable,booktabs,array}
\usepackage{calc} % for calculating minipage widths
% Correct order of tables after \paragraph or \subparagraph
\usepackage{etoolbox}
\makeatletter
\patchcmd\longtable{\par}{\if@noskipsec\mbox{}\fi\par}{}{}
\makeatother
% Allow footnotes in longtable head/foot
\IfFileExists{footnotehyper.sty}{\usepackage{footnotehyper}}{\usepackage{footnote}}
\makesavenoteenv{longtable}
\usepackage{graphicx}
\makeatletter
\newsavebox\pandoc@box
\newcommand*\pandocbounded[1]{% scales image to fit in text height/width
  \sbox\pandoc@box{#1}%
  \Gscale@div\@tempa{\textheight}{\dimexpr\ht\pandoc@box+\dp\pandoc@box\relax}%
  \Gscale@div\@tempb{\linewidth}{\wd\pandoc@box}%
  \ifdim\@tempb\p@<\@tempa\p@\let\@tempa\@tempb\fi% select the smaller of both
  \ifdim\@tempa\p@<\p@\scalebox{\@tempa}{\usebox\pandoc@box}%
  \else\usebox{\pandoc@box}%
  \fi%
}
% Set default figure placement to htbp
\def\fps@figure{htbp}
\makeatother





\setlength{\emergencystretch}{3em} % prevent overfull lines

\providecommand{\tightlist}{%
  \setlength{\itemsep}{0pt}\setlength{\parskip}{0pt}}



 


\KOMAoption{captions}{tableheading}
\makeatletter
\@ifpackageloaded{caption}{}{\usepackage{caption}}
\AtBeginDocument{%
\ifdefined\contentsname
  \renewcommand*\contentsname{Table of contents}
\else
  \newcommand\contentsname{Table of contents}
\fi
\ifdefined\listfigurename
  \renewcommand*\listfigurename{List of Figures}
\else
  \newcommand\listfigurename{List of Figures}
\fi
\ifdefined\listtablename
  \renewcommand*\listtablename{List of Tables}
\else
  \newcommand\listtablename{List of Tables}
\fi
\ifdefined\figurename
  \renewcommand*\figurename{Figure}
\else
  \newcommand\figurename{Figure}
\fi
\ifdefined\tablename
  \renewcommand*\tablename{Table}
\else
  \newcommand\tablename{Table}
\fi
}
\@ifpackageloaded{float}{}{\usepackage{float}}
\floatstyle{ruled}
\@ifundefined{c@chapter}{\newfloat{codelisting}{h}{lop}}{\newfloat{codelisting}{h}{lop}[chapter]}
\floatname{codelisting}{Listing}
\newcommand*\listoflistings{\listof{codelisting}{List of Listings}}
\makeatother
\makeatletter
\makeatother
\makeatletter
\@ifpackageloaded{caption}{}{\usepackage{caption}}
\@ifpackageloaded{subcaption}{}{\usepackage{subcaption}}
\makeatother
\usepackage{bookmark}
\IfFileExists{xurl.sty}{\usepackage{xurl}}{} % add URL line breaks if available
\urlstyle{same}
\hypersetup{
  pdftitle={Tarjetas de Credito y Pago Minimo},
  pdfauthor={MIguel Sandoval, Juan Carrillo, Aldo Hernandez y Rosendo Hernandez},
  colorlinks=true,
  linkcolor={blue},
  filecolor={Maroon},
  citecolor={Blue},
  urlcolor={Blue},
  pdfcreator={LaTeX via pandoc}}


\title{Tarjetas de Credito y Pago Minimo}
\author{MIguel Sandoval, Juan Carrillo, Aldo Hernandez y Rosendo
Hernandez}
\date{}
\begin{document}
\maketitle


\subsection{Introducción a las Tarjetas de
Crédito💳}\label{introducciuxf3n-a-las-tarjetas-de-cruxe9dito}

La tarjeta de crédito es un instrumento financiero que consiste en una
tarjeta de plástico con un microchip en la parte trasera, mediante la
misma el banco otorga a sus clientes una línea de crédito por una cierta
cantidad. Las tarjetas de crédito en México surgen cuando el Banco
Nacional de México decide lanzar al mercado una tarjeta de crédito en
1968.

\subsection{Cuota de manejo🪙}\label{cuota-de-manejo}

La cuota de manejo es un monto que se cobra periódicamente por el uso y
administración de la tarjeta, se trata de una tarifa fija que aplica el
banco por distintos servicios que el mismo te pueda proveer, según la
tarjeta que tengas. En la mayoría de los casos se cobra de manera
mensual, pero esto puede variar por la institución bancaria. Aunque
existen tarjetas que promocionan exenciones temporales o condiciones
especiales, el cobro de cuota de manejo es un estándar en la industria
bancaria.

\subsection{Comisiones🧾}\label{comisiones}

Todas las tarjetas de crédito cuentan con un apartado de comisiones,
estas son cargos que el banco o institución financiera aplica por la
administración de la tarjeta, el uso de sus servicios, o por
transacciones específicas. Las más comunes son la comisión anual o de
administración para cubrir los costos de la tarjeta y sus beneficios, la
comisión por disposición de efectivo al retirar dinero y la comisión por
pago tardío, pero también existen otras como la de emisión,
mantenimiento o por transferencia de saldo.

\subsection{Limite de Credito❗}\label{limite-de-credito}

Un límite de crédito es el monto máximo de dinero que un prestamista
(como un banco o una institución financiera) te otorga para usar en una
cuenta de crédito. Como una linea de crédito se basa en tu historial y
capacidad de pago, y se incrementa a medida que realizas compras.

Las instituciones financieras evalúan varios factores para determinar tu
límite de crédito como:

➡Historial crediticio\\
Es el registro detallado de como a manejado sus obligaciones
financieras, prestamos o hipotecas

➡Nivel de ingresos\\
Indica la capacitad de ingresos de una persona o entidad en un periodo
determinado

\subsection{Fecha de Corte📆}\label{fecha-de-corte}

Es el día en que el banco finaliza un ciclo de facturación y genera el
estado de cuenta, acumulando todas las compras y transacciones
realizadas hasta esa fecha.

Todas las operaciones realizadas después de la fecha de corte se
registrarán en el día siguiente de la fecha de corte y marca el final de
un período de aproximadamente 30 días de uso de la tarjeta.

\subsection{Pago Total de la Tarjeta⚖️}\label{pago-total-de-la-tarjeta}

El monto total que debes pagar antes de la fecha límite para evitar
cargos por intereses , con esto evitar el cobro de intereses ordinarios.
Al liquidar este saldo completo, tus compras se cancelan y no se genera
ningún interés sobre el gasto del periodo. Es la mejor práctica para
usar tu tarjeta y mantener una buena salud financiera, según sea la
banca de tu agrado.

🟢Beneficios:

➡Evitas pagar intereses por tus compras.\\
➡Mantienes el saldo de tu tarjeta en cero, como si hubieras usado
efectivo.\\
➡Construyes un buen historial crediticio.

\subsection{Pago Minimo📉}\label{pago-minimo}

Cuando no se puede pagar el saldo total en la fecha límite de pago,
entonces deberá realizar cuando menos el pago mínimo.

El pago mínimo es la cantidad más pequeña que se paga por un adeudo que
se tenga en tarjeta de crédito.\\
Pero el pago mínimo y el pago para no generar intereses son conceptos
distintos:

\begin{itemize}
\item
  \textbf{\emph{Pago Mínimo}}💸\\
  Es el requisito mínimo para evitar multas y reportes negativos, pero
  SÍ genera intereses sobre el saldo restante.
\item
  \textbf{\emph{Pago para NO Generar Intereses}}💰\\
  Si pagas el saldo total que aparece en tu estado de cuenta antes de la
  fecha límite, la institución financiera no te cobrará ningún interés.
  Esta es la manera más inteligente y económica de utilizar una tarjeta
  de crédito.
\end{itemize}

\subsection{Intereses💹}\label{intereses}

Los intereses son el costo que pagas al banco o institución financiera
cuando no liquidas el total de tu deuda en la fecha de corte o límite de
pago.

Las tarjetas de credito utilizan una tasa de interes anual, y
normalmente en pero en la mayoria de los casos se divide esta tasa para
cobrar de manera mensual o incluso diaria.

Tipos de interés en tarjetas de crédito:

\begin{itemize}
\item
  \textbf{\emph{Intereses ordinarios}}\\
  Los intereses ordinarios se generan cuando no pagas el total de tu
  saldo despues de la fecha de corte.
\item
  \textbf{\emph{Intereses moratorios}}\\
  Los intereses moratorios son una penalización por no realizar tu pago
  a tiempo y por no realizar tu pago minimo en la fecha limite de pago;
  estos son más altos que los ordinarios y se aplican a los saldos
  vencidos desde el momento del retraso. Estos llegan a afectar tu
  historial acrediticio.
\end{itemize}

\subsection{Costo Anual Total💲}\label{costo-anual-total}

Es un porcentaje que representa el costo total de un crédito al incluir
todos los intereses, comisiones y gastos asociados durante un año. Este
indicador, creado por el Banco de México para fines informativos,
permite comparar de manera estandarizada los productos crediticios y
saber cuál de ellos genera un menor o mayor costo financiero para el
consumidor.

\begin{itemize}
\tightlist
\item
  \textbf{\emph{Componentes del CAT}}
\end{itemize}

➡Tasa de interes\\
➡Comisiones\\
➡Otros cargos

\subsection{Descargar presentación}\label{descargar-presentaciuxf3n}

\href{equipo2.pdf}{Descargar PDF de las diapositivas}




\end{document}
